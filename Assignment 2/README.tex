\documentclass[12pt]{article}
\usepackage{ragged2e} % load the package for justification
\usepackage{hyperref}
\usepackage[utf8]{inputenc}
\usepackage{pgfplots}
\usepackage{tikz}
\usetikzlibrary{fadings}
\usepackage{filecontents}
\usepackage{multirow}
\usepackage{amsmath}
\pgfplotsset{width=10cm,compat=1.17}
\setlength{\parskip}{0.75em} % Set the space between paragraphs
\usepackage{setspace}
\setstretch{1.2} % Adjust the value as per your preference
\usepackage[margin=2cm]{geometry} % Adjust the margin
\setlength{\parindent}{0pt} % Adjust the value for starting paragraph
\usetikzlibrary{arrows.meta}
\usepackage{mdframed}
\usepackage{float}

\usepackage{hyperref}

% to remove the hyperline rectangle
\hypersetup{
	colorlinks=true,
	linkcolor=black,
	urlcolor=blue
}

\usepackage{xcolor}
\usepackage{titlesec}
\usepackage{titletoc}
\usepackage{listings}
\usepackage{tcolorbox}
\usepackage{lipsum} % Example text package
\usepackage{fancyhdr} % Package for customizing headers and footers



% Define the orange color
\definecolor{myorange}{RGB}{255,65,0}
% Define a new color for "cherry" (dark red)
\definecolor{cherry}{RGB}{148,0,25}
\definecolor{codegreen}{rgb}{0,0.6,0}



%%%%%%%%%%%%%%%%%%%%%%%%%%%%%%%%%%%%%%%%%%%%%%%%%%%%%%%%%%%%%%%%%%%%%
% Apply the custom footer to all pages
\pagestyle{fancy}

% Redefine the header format
\fancyhead{}
\fancyhead[R]{\textcolor{orange!80!black}{\itshape\leftmark}}

\fancyhead[L]{\textcolor{black}{\thepage}}


% Redefine the footer format with a line before each footnote
\fancyfoot{}
\fancyfoot[C]{\footnotesize P. Pasandide, McMaster University, Programming for Mechatronics - MECHTRON 2MP3. \footnoterule}

% Redefine the footnote rule
\renewcommand{\footnoterule}{\vspace*{-3pt}\noindent\rule{0.0\columnwidth}{0.4pt}\vspace*{2.6pt}}

% Set the header rule color to orange
\renewcommand{\headrule}{\color{orange!80!black}\hrule width\headwidth height\headrulewidth \vskip-\headrulewidth}

% Set the footer rule color to orange (optional)
\renewcommand{\footrule}{\color{black}\hrule width\headwidth height\headrulewidth \vskip-\headrulewidth}

%%%%%%%%%%%%%%%%%%%%%%%%%%%%%%%%%%%%%%%%%%%%%%%%%%%%%%%%%%%%%%%%%%%%%


% Set the color for the section headings
\titleformat{\section}
{\normalfont\Large\bfseries\color{orange!80!black}}{\thesection}{1em}{}

% Set the color for the subsection headings
\titleformat{\subsection}
{\normalfont\large\bfseries\color{orange!80!black}}{\thesubsection}{1em}{}

% Set the color for the subsubsection headings
\titleformat{\subsubsection}
{\normalfont\normalsize\bfseries\color{orange!80!black}}{\thesubsubsection}{1em}{}


%%%%%%%%%%%%%%%%%%%%%%%%%%%%%%%%%%%%%%%%%%%%%%%%%%%%%%%%%%%%%%%%%%%%%
% Set the color for the table of contents
\titlecontents{section}
[1.5em]{\color{orange!80!black}}
{\contentslabel{1.5em}}
{}{\titlerule*[0.5pc]{.}\contentspage}

% Set the color for the subsections in the table of contents
\titlecontents{subsection}
[3.8em]{\color{orange!80!black}}
{\contentslabel{2.3em}}
{}{\titlerule*[0.5pc]{.}\contentspage}

% Set the color for the subsubsections in the table of contents
\titlecontents{subsubsection}
[6em]{\color{orange!80!black}}
{\contentslabel{3em}}
{}{\titlerule*[0.5pc]{.}\contentspage}


%%%%%%%%%%%%%%%%%%%%%%%%%%%%%%%%%%%%%%%%%%%%%%%%%%%%%%%%%%%%%%%%%%%%%
% set a format for the codes inside a box with C format
\lstset{
	language=C,
	basicstyle=\ttfamily,
	backgroundcolor=\color{blue!5},
	keywordstyle=\color{blue},
	commentstyle=\color{codegreen},
	stringstyle=\color{red},
	showstringspaces=false,
	breaklines=true,
	frame=single,
	rulecolor=\color{lightgray!35}, % Set the color of the frame
	numbers=none,
	numberstyle=\tiny,
	numbersep=5pt,
	tabsize=1,
	morekeywords={include},
	alsoletter={\#},
	otherkeywords={\#}
}




%\input listings.tex



% Define a command for inline code snippets with a colored and rounded box
\newtcbox{\codebox}[1][gray]{on line, boxrule=0.2pt, colback=blue!5, colframe=#1, fontupper=\color{cherry}\ttfamily, arc=2pt, boxsep=0pt, left=2pt, right=2pt, top=3pt, bottom=2pt}




\tikzset{%
	every neuron/.style={
		circle,
		draw,
		minimum size=1cm
	},
	neuron missing/.style={
		draw=none, 
		scale=4,
		text height=0.333cm,
		execute at begin node=\color{black}$\vdots$
	},
}



%%%%%%%%%%%%%%%%%%%%%%%%%%%%%%%%%%%%%%%%%%%%%%%%%%%%%%%%%%%%%%%%%%%%%

% Define a new tcolorbox style with default options
\tcbset{
	myboxstyle/.style={
		colback=orange!10,
		colframe=orange!80!black,
	}
}

% Define a new tcolorbox style with default options to print the output with terminal style


\tcbset{
	myboxstyleTerminal/.style={
		colback=blue!5,
		frame empty, % Set frame to empty to remove the fram
	}
}

\mdfdefinestyle{myboxstyleTerminal1}{
	backgroundcolor=blue!5,
	hidealllines=true, % Remove all lines (frame)
	leftline=false,     % Add a left line
}


\begin{document}
	
	\justifying
	
	\begin{center}
		\textbf{{\large <Subject>}}
		
		\textbf{Developing a Basic Genetic Optimization Algorithm in C} 
		
		<your name>
	\end{center}
	

	
	
	
	\section{Introduction}
	
	
	\section{Programming Genetic Algorithm}

	\subsection{Implementation (12 points)}
	
	Files you have downloaded are:
	
	\begin{itemize}
		\item \codebox{OF.c} (Do not change anything in it)
		\item \codebox{functions.h} (Do not change anything in it)
		\item \codebox{GA.c}
		\item \codebox{functions.c}
	\end{itemize}

		
		\begin{lstlisting}
			#include <math.h>
			#include "functions.h"
			
			double Objective_function(int NUM_VARIABLES, double x[NUM_VARIABLES])
			{
				double sum1 = 0.0, sum2 = 0.0;
				for (int i = 0; i < NUM_VARIABLES; i++)
				{
					sum1 += x[i] * x[i];
					sum2 += cos(2.0 * M_PI * x[i]);
				}
				return -20.0 * exp(-0.2 * sqrt(sum1 / NUM_VARIABLES)) - exp(sum2 / NUM_VARIABLES) + 20.0 + M_E;
			}
		\end{lstlisting}
	
	

	
	
	The number of decision variables in the optimization problem is set to 2, with upper and lower bounds for both decision variables set to +5 and -5, respectively:
	
	\begin{lstlisting}
		int NUM_VARIABLES = 2;
		double Lbound[] = {-5.0, -5.0};
		double Ubound[] = {5.0, 5.0};
	\end{lstlisting}
	
	
	\begin{mdframed}[style=myboxstyleTerminal1]
		\begin{verbatim}
			Genetic Algorithm is initiated.
			------------------------------------------------
			The number of variables: 2
			Lower bounds: [-5.000000, -5.000000]
			Upper bounds: [5.000000, 5.000000]
			
			Population Size:   100
			Max Generations:   10000
			Crossover Rate:    0.500000
			Mutation Rate:     0.100000
			Stopping criteria: 0.0000000000000001
			
			Results
			------------------------------------------------
			CPU time: 1.143935 seconds
			Best solution found: (-0.0001323239496775, -0.0000225231135396)
			Best fitness: 0.0003801313729501
		\end{verbatim}
	\end{mdframed}
	
	 
	\begin{table}[h!]
		\caption{Results with Crossover Rate = 0.5 and Mutation Rate = 0.05}
		\label{table:1}
		\centering
		\begin{tabular}{c c c c c c}
			\hline
			Pop Size & Max Gen & \multicolumn{3}{c}{Best Solution} & CPU time (Sec) \\
			& & $x_1$ & $x_2$ & Fitness & \\
			\hline
			10  & 100    &  &  & &\\
			100 & 100    &  &  & &\\
			1000& 100    &  &   & &\\
			10000& 100    &  &   & &\\
			\hline
			1000  & 1000   &  & & &\\
			1000 & 10000  &  &  & &\\
			1000& 100000 &  &   & &\\
			1000& 1000000 &  &   & &\\
			\hline
		\end{tabular}
	\end{table}

	\begin{table}[h!]
		\caption{Results with Crossover Rate = 0.5 and Mutation Rate = 0.2}
		\label{table:1}
		\centering
		\begin{tabular}{c c c c c c}
			\hline
			Pop Size & Max Gen & \multicolumn{3}{c}{Best Solution} & CPU time (Sec) \\
			& & $x_1$ & $x_2$ & Fitness & \\
			\hline
			10  & 100    &  &  & &\\
			100 & 100    &  &  & &\\
			1000& 100    &  &   & &\\
			10000& 100    &  &   & &\\
			\hline
			1000  & 1000   &  & & &\\
			1000 & 10000  &  &  & &\\
			1000& 100000 &  &   & &\\
			1000& 1000000 &  &   & &\\
			\hline
		\end{tabular}
	\end{table}
	
	
	\subsection{Report and \codebox{Makefile} (3 points)}
	
	
	
	\subsection{Improving the Performance - Bonus (+1 points)}
	
	
	
	
	\subsection{Bonus (+3 points) - Only the fastest program!}
	
	

	 
	
	\section{Appendix}
	
	
\end{document}